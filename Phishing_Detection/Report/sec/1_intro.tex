\section{Introduction}
Phishing is the practice of sending fraudulent communications that appear to come from a reputable source. The goal is to steal sensitive data such as credit card and login information or to install malware on the victim’s machine. Hackers do not need to crack any complex cipher code nor breach a hard firewall. Instead, they simply send emotional, critical, or sensible e-mails urging recipients to introduce their personal credentials by clicking on a link. Recipients are then redirected to fake web that look very similar to those targeted authentic websites.\\
During the pandemic, there was a notable increase in phishing activities. This increase can be attributed to the increased demand for current information during this period, which presented an opportunity for malicious actors to impersonate trusted entities, such as health services and government websites.\\
To get a better understanding,, let us consider an incident in the IIT Hyderabad campus:\\
A simulation attack was conducted by the Computer Center, IIT Hyderabad in collaboration with Kludge, networking club, IIT Hyderabad as part of their Cybersecurity campaign.
Their mode of attack was phishing emails. The subjects of
the above mentioned phishing emails were:\\
 \begin{itemize}
     \item[-]WiFi Password Expiration Notification.
     \item[-]Avail the Chat GPT Plus Subscription.
     \item[-]Reviewing the APAR Appraisal Form.
     \item[-]Additional Funds Allocated to Your Department.
     \item[-]Urgent Report Submission.
 \end{itemize}
 These emails were sent from the domain “iith-ac.c-0m.com”.\\
 The use of special characters in the above domain and its resemblance to an existing domain is an indication for it to be a phishing website. Similarly, the dataset includes approximately 87 features derived from the URL, its content, and external records to determine its legitimacy.\\
Thus, Machine learning could be used for classifying phishing websites as it offers adaptability to evolving attack methods, enables automated and large-scale analysis of datasets, excels in pattern recognition, provides real-time
detection, reduces false positives, enhances accuracy through continuous learning, and allows for behavioral analysis, all of which contribute to effective and efficient identification of phishing threats.